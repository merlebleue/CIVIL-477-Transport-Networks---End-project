\documentclass{EESD}
% To change the slides size go to EESD.cls file and edit the preamble as explained.

% ---- Add your Meta-data to the PDF (Copyrights Kinda!) ----
\hypersetup{
  pdfinfo={
    Title={Light Rail and Park\&Ride facilities in Sioux Falls},
    Author={Julien Ars},
  }
}

% Important packages to be called
\usepackage{subcaption} % for adding sub-figures
\usepackage{graphicx}
\usepackage{tikz} % for cool graphics and drawings

\usepackage[absolute,overlay]{textpos} % To place the figures by coordinates (x,y) - Beamer doesn't support floats XD
\usepackage{multicol} % To adjust items and stuff automatically in a number of a pre-specified columns
\graphicspath{{Figures/}}
\usepackage[utf8]{inputenc}
\usepackage{amsmath}
\usepackage{amsfonts}
\usepackage{amssymb}
\usepackage{lipsum} % Just a dummy text generator
\usepackage{hyperref}
% fonts packages
\usepackage{ragged2e} % Justified typesetting

% ChatGPT for table
\usepackage{booktabs}
\usepackage{multirow}
\usepackage{adjustbox}
\usepackage{pifont}
\usepackage{array}

\usepackage{algorithm}
\usepackage{algpseudocode}
\usepackage{xcolor}

% For References Only
\usepackage[style=authortitle,backend=bibtex]{biblatex}
\addbibresource{references.bib} % Call the references database
\AtBeginBibliography{\tiny} % Specify font size (Size matters)
\renewcommand{\footnotesize}{\tiny}

% For adding code blocks
\usepackage{listings}
\lstset
{
    language=[LaTeX]TeX,
    breaklines=true,
    basicstyle=\tt\scriptsize,
    keywordstyle=\color{blue},
    identifierstyle=\color{magenta},
    commentstyle=\color{red},
    rulecolor=\color{black},
    numbers=left,
    numberstyle=\tiny\color{black},
    % framexleftmargin=15pt,
    frame = single,
}

\author{Julien Ars}
\title[Light rail and P\&R facilities in Sioux Falls]{Light Rail and Park\&Ride facilities in Sioux Falls}

\institute[ENAC]{{\'Ecole Polytechnique F\'ed\'erale de Lausanne (EPFL)}{\newline\newline Civil Engineering}}
\subject{Talk}
\date{2025-05-28}

\begin{document}

{ % <-- don't forget the scope resolution (Brackets) to limit the background change only for the cover page; otherwise, it will override the whole document's background :)
\usebackgroundtemplate{} % To add a background for this slide XD - Empty one
\coverpage{
\titlepage{~}
% To add additional text to the title components 
{\newline \small CIVIL-477 - Transport Networks Modelling \& Analysis}
}
} % <-- and yeah close them too :)


\setbeamertemplate{logo}{} % To override the logo from the other slides and delete it completely

% Use smart division for the TOC
\section*{Table  of content}
\begin{frame}{Outlines}
\tableofcontents
\end{frame}

% ----------------------- SiouxFalls
\section*{A quick note}
\begin{frame}{My projects this semester}
	\begin{itemize}
		\item \textbf{A real-time heuristic for the railway timetable rescheduling problem} (Semester project : 8 ECTS) \\
		Under the supervision of Lea Ricard and Negar Rezvany \\
		Develop an efficient heuristic to be used in delay management in the RER Vaud
		\textcolor{bg!75!normal text.fg}{\item  \textbf{MaxSMT in delay management} (Pre-study : 3 ECTS) \\
		Under the supervision of Thomas Dubach (ETH - IVT) \\
		Use an existing 'MaxSMT' solver to limit impact of a delay in time and space}
	\end{itemize}
\end{frame}



% -----------------------Introduction
\section{Introduction}
\breakingframe{
\begin{textblock*}{0.8\textwidth}[0,0.5](0.17\textwidth,  0.55\textheight)
\centering \Huge\textbf{\textcolor{black}{Introduction}}
\end{textblock*}
}

\subsection{Problem definition}
\begin{frame}{Problem definition}
		
	\begin{itemize}
		\item Given a railway network with stations $s \in S$ each with a set of platforms $P_s$, bifurcations $b \in B$, section tracks $q \in Q$ between stations and/or bifurcations.
		\item Given a planned schedule, with predefined trains $k \in K$ composed of a sequence of stations and bifurcations with given arrival and departure times.
		\item Given passengers groups $g \in G$, each with a desired origin station $o_g$, destination station $d_g$ and desired departure time $t_g$ (and size $n_g$).
		\item Given a disruption with known duration, impacting one or multiple section tracks, partially (reduced speed) or completely.
	\end{itemize}

	Get a new schedule for the train that minimises the passenger inconvenience $f_p$, the operational costs $f_o$ and the deviation from the original schedule.
\end{frame}


\subsection{Litterature}
\begin{frame}{Relevant litterature for the project}
	\begin{enumerate}
		\item Work by Stefan Binder (2017) : \begin{itemize}
			\item \textbf{\cite{BINDER_main}} : \\
				MILP definition in a time-indexed formulation for a similar problem. \\
				{\it We reuse their instance (simplified Dutch network) for testing.\\
				We reuse their passenger cost definition.}
			\item \textbf{\cite{BINDER_priority}} : \\
				Proposition of an exogenous ordering of the passengers (instead of endogenous) \\
				{\it We consider using a similar approach, and compare our performance with the proposed algorithm for passenger assignment}
		\end{itemize}
	\end{enumerate}
\end{frame}

\begin{frame}{Relevant litterature for the project}
	\begin{enumerate}
		\item Work by Stefan Binder (2017) : \begin{itemize}
			\item ...
			\item \textbf{\cite{BINDER_heuristic}} : \\
				Proposed ALNS heuristic for the problem \\
				{\it Starting point for our heuristic}
		\end{itemize}
		\item Also on ALNS : \begin{itemize}
				\item \textbf{\cite{ALNS}} : \\
				First introduction of an {\it Adaptative Large Neighborhood Search} (ALNS) heuristic
			\end{itemize}
	\end{enumerate}
\end{frame}

\begin{frame}{Relevant litterature for the project}
	\begin{enumerate}[3]
		\item New developments :
	\end{enumerate}

	\begin{table}
		\vspace{-5pt}
	\centering
	\begin{adjustbox}{width=\textwidth}
	\scriptsize
	\begin{tabular}{@{}llclccccc>{\raggedright\arraybackslash}p{3.5cm}@{}}
	\toprule
	\textbf{Author(s)} & \textbf{Net.} & \textbf{Algo.} & \textbf{Mult. obj.} & \textbf{Recovery decisions} & \textbf{Pass. assign.} & \textbf{Tracks} & \textbf{Seat reser.} & \textbf{Veh. capa.} & \textbf{Largest instance size} \\ 
	\midrule
	Veelenturf et al. (2016) & E-A & MILP & \ding{51} & C, D, O, R, ST & None & Multiple & & & Network, S = 26, T = 165, D = 2h, H = 2.75h \\
	Binder et al. (2017)     & S-T     & MILP & \ding{51} & A, C, D, E, O, R & Dynamic & Double & & & Network, S = 11, T = 24, D = 2h, H = 2h \\
	Zhu \& Goverde (2019)     & E-A & MILP & & A, C, D & Static & Single, double & & & Network, S = 17, T = 24/h, D = 3h, H = 4h \\
	Altazin et al. (2020)    & E-A & SIM, H & \ding{51} & A, C, ST & Dynamic & Single, double & & & Network, S = 29, T = 25, D = 20 min, H = 2.5 \\
	Zhu \& Goverde (2020b)    & E-A & MH & \ding{51} & A, C, D, O, ST & & & & & Network, S = 17, T = 72, D = 2h, H = 3h \\
	Hong et al. (2021)       & S-T     & MILP & \ding{51} & C, D, O, R & Dynamic & Double & \ding{51} & \ding{51} & Line, S = 8, T = 20, D = 55 min, H = 4h \\
	Zhan et al. (2021)       & S-T     & ADMM, DP & \ding{51} & C, D, O, R & Dynamic & Double & \ding{51} & \ding{51} & Network, S = 17, T = 20, D = 2h, H = 6h \\
	Zhang et al. (2023)      & S-T     & LR-H & \ding{51} & C, D, O, R & Static & Double & \ding{51} & \ding{51} & Network, S = 47, T = 350, D = 6h, H = 18h \\
	Xiu et al. (2024)        & S-T     & ADMM & \ding{51} & C, D, O, R & Static & Double & \ding{51} & \ding{51} & Network, S = 15, T = 36, D = 2h, H = 6.5h \\
	\midrule
	Our project 			 & E-A 	   & MILP, H & \ding{51} & A, C, D, E, O, R, ST & Dynamic & Multiple & & \ding{51} & ? \\
	\bottomrule
	\end{tabular}
	\end{adjustbox}
	\caption{Characteristics of selected railway rescheduling problems \footcite{Lea_litterature}}
	\end{table}
	\vspace{-0.5pt}
\end{frame}


\subsection{Model formulation}
\begin{frame}{Model formulation : Graph}
	\begin{itemize}
		\item Event - Activity Graph, with two layers (trains / passengers)
		\item An event (= one node) correspond to a train arriving or leaving at a given station or bifurcation.
		\item Activities (= arc) link the events. There are multiple types :
	\end{itemize}
	\begin{table}
		\centering
		\begin{tabular}{cc}
			\textbf{Train activities} & \textbf{Passenger activities} \\
			\cmidrule{1-1}\cmidrule{2-2}
			Running & Running \\
			Waiting & Dwelling \\
			Pass-through & Transfer\\
			Starting & Access \\
			Ending & Egress \\
			Short-turning & Penalty \\
		\end{tabular}
	\end{table}
	Train activities are detailed with platforms at stations and tracks on the section.\\
	Passenger activities are aggregated.
\end{frame}

\begin{frame}{Model formulation : Costs}
	Three objective functions ($\epsilon$-constraint method):
	\begin{enumerate}
		\item Passenger costs (see table)
		\item Operation costs: $\sim$ the number of kilometers trains run 
		\item Deviation costs
	\end{enumerate}
	\begin{table}
		\centering
		\begin{tabular}{cc}
			\textbf{Passenger activities} & \textbf{Costs} \\
			\midrule
			Running & $y_j^k - y_i^k$\\
			Dwelling & $\beta_1 (y_j^k - y_i^k$)\\
			Transfer& $\beta_2 + (y_j^k - y_i^k$)\\
			Access & $\beta_3 \max(0, t_g - y_i^k) + \beta_4 \max(0, y_i^k - t_g)$\\
			Egress & $0$\\
			Penalty & $5 * H = 600$\\
		\end{tabular}
	\end{table}
	$$\beta_1 = 2.5 \quad \beta_2 = 10 \quad \beta_3 = 0.5 \quad \beta_4 = 1$$
\end{frame}

\subsection{Motivation for an heuristic}
\begin{frame}{Motivation for an heuristic : RER Vaud}
	\begin{itemize}
		\item The MILP run fast on the dutch instance ($\sim$ 20 s for optimizing), but RER Vaud is much bigger.
		\item Currently, the passenger layer would be too big for inclusion in Gurobi.
	\end{itemize}
\end{frame}


 
% ----------------------- Passenger assignment
\section{Passenger assignment}
\breakingframe{
\begin{textblock*}{0.8\textwidth}[0,0.5](0.17\textwidth,  0.55\textheight)
\centering \Huge\textbf{\textcolor{black}{Passenger assignment}}
\end{textblock*}
}


\subsection{Performance}
\begin{frame}{Performance : Setup}
	We will compare our algorithm to the results of the MIP for the passenger assignment.
	This will be done on the dutch network, with the initial timetable (without disruption) that includes 15 trains. 
	We will compare the calculation time with different train capacities (each time for 55 groups):
	\begin{itemize}
		\item With a train capacity of 400 (which correspond to 4 groups), we have no capacity conflict
		\item With a train capacity of 300, we have some capacity conflict
		\item With a train capacity of 200, we have many capacity conflicts
	\end{itemize}
\end{frame}
\begin{frame}{Performance : Computation time}
	\begin{table}
		\centering
		\begin{tabular}{l|ccc}
			\toprule 
			Train capacity & 200 & 300 & 400 \\
			\midrule
			Gurobi (MILP) & 0.853 s & 1.000 s & 0.712 s\\
			Dynamic assignment (priorities) & 0.225 s & 0.188 s & 0.152 s\\
			Dynamic assignment (presence in the train) & 0.204 s & 0.192 s & 0.150 s\\
			\bottomrule
		\end{tabular}
		\caption{Computation time for our algorithm and the MILP}
	\end{table}
	\begin{itemize}
		\item Running the shortest path on each o-d pair was taking $\sim$ 0.2 s.
		\item Our algorithm outperforms Gurobi at all capacities.
		\item Depending on the use of Dijkstra, we even improve on Binder's algorithm if the network is not nearly full.
	\end{itemize}
\end{frame}
\begin{frame}{Performance : Passenger cost}
	\begin{table}
		\centering
		\begin{tabular}{l|ccc}
			\toprule 
			Train capacity & 200 & 300 & 400 \\
			\midrule
			Gurobi (MILP) & 920050 & 499000 & 294350\\
			Dynamic assignment (priorities) & 1005150 & 518950 & 294350\\
			Dynamic assignment (presence in the train) & 1121250 & 528600 & 294350\\
			\bottomrule
		\end{tabular}
		\caption{Passenger cost results for our algorithm and the MILP}
	\end{table}
	\begin{itemize}
		\item When capacity constraints are not reached, the results are the same \ding{51}
		\item Cost increase when reducing capacity (\ding{51}), at 200 capacity it is enormous : many use of penalty arcs
		\item Gurobi has lowest cost when capacity constraints are reached : System Optimum
		\item Considering the presence in the train increase the cost (more constrained, realistic description)
	\end{itemize}
\end{frame}


% ----------------------- Next steps
\section{Next steps}
\breakingframe{
\begin{textblock*}{0.8\textwidth}[0,0.5](0.17\textwidth,  0.55\textheight)
\centering \Huge\textbf{\textcolor{black}{Next steps}}
\end{textblock*}
}
\begin{frame}{Next steps}
	Mandatory steps :
	\begin{enumerate}
		\item Develop a dynamic validation of a solution
		\item Implement an ALNS heuristic
	\end{enumerate}
	Facultative steps :
	\begin{itemize}
		\item Try another shortest path algorithm / implementation
		\item Try solving for system optimum dynamically \begin{itemize}
			\item Use as a lower bound to speed up Gurobi
		\end{itemize}
	\end{itemize}

\end{frame}




\section*{This is the end}
\breakingframe{
\begin{textblock*}{0.8\textwidth}[0,0.5](0.17\textwidth,  0.55\textheight)
\centering \Huge\textbf{\textcolor{black}{Thank you for your attention}}
\end{textblock*}
}

% -----------------------References
\section{Bibliography}
% \begin{frame}[allowframebreaks]{\\References}\vspace{4pt}
\begin{frame}{References}\vspace{4pt}
\tiny{\printbibliography}
\end{frame}
\normalsize

\end{document}