\section{Conclusion}

This study evaluated the impact of introducing a light rail network and park-and-ride (P\&R) facilities on the Sioux Falls road network. The results demonstrate that P\&R facilities not only improve overall system performance — measured by total travel time — but can also allow a more compact network to achieve similar or even greater efficiency compared to a larger network without P\&R. Our findings suggest that strategically placing P\&R facilities at selected stations (we propose nodes 4, 14, 15, and 21) and focusing the light rail network on the central core could provide the best balance between cost and effectiveness.

Several limitations of the current analysis should be addressed in future work. The model only considers travel time, neglecting factors such as waiting time, transfer penalties, parking and ticket costs, and capacity constraints. Incorporating these elements - by adding costs to onboarding and offboarding links, modeling transfers between lines, and introducing capacity limits - would yield a more realistic assessment and help determine operational requirements such as headways and rolling stock needs.

Finally, the assumption that users always choose the route with the lowest modeled travel time may oversimplify real-world behavior. Incorporating more advanced mode choice models, such as logit models, could better capture user preferences and variability, and better represent the real impact of adding a public transport infrastructure.

\vfill

\section*{Disclosure}

This work was completed independently and reflects my own understanding and effort. To improve clarity and correctness of language, I used large language model (LLM)-based tools for grammar and style suggestions.

All code and results produced for this project are available on GitHub: \url{https://github.com/merlebleue/CIVIL-477-Transport-Networks---End-project}