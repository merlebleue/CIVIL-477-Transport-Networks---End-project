\section{Conclusion}

This study studied the impact of adding a Light Rail network on the city of Sioux Falls, and whether adding Park-and-Ride facilities at the stations would increase its impact. We showed that Park-and-Ride facilities helped improving the performance of the system (in terms of total travel time), while additionally suggesting they could help make a smaller network as efficient if not more efficient. We identify next steps in the analysis : reducing the number of stations with P\&R (we propose nodes 4, 14, 15 and 21) and reducing the network length to the central core. Studying these new scenarios could help provide the best light rail network at a good price over quality ratio.

Also, the precision of our analysis could be strenghtened in multiple ways. In this study, we made the assumption that we could neglect any other factor than the travel time, and we even did not take into account the waiting time or the cost of transfers. This was strong assumptions, and by adding costs to the on-boarding and off-boarding links, as well as dividing the tram network according to the lines and adding transfer links between them, we could model it better. Adding capacity constraints to the light rail is another step that could be taken to identify headway and capacity requirements, which would guide the rolling stock invesment cost.

Finally, modelling mode choice is a complex phenomenon, and assuming users always take the best choice according to our modelling could be a very strong assumption. Logit models, for example could be a better option.

\vfill

\section*{Disclosure}

This work was completed independently and reflects my own understanding and effort. To improve clarity and correctness of language, I used large language model (LLM)-based tools for grammar and style suggestions.

Any code and results produced for this project are availlable on Github : \url{https://github.com/merlebleue/CIVIL-477-Transport-Networks---End-project}